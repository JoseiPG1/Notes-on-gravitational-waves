% type of document
\documentclass{article}
% new section new page
\usepackage{titlesec}
\newcommand{\sectionbreak}{\clearpage}
% hyper-references
\usepackage{hyperref}
% align equation
\usepackage{amsmath}
% figures such as square
\usepackage{amssymb}
% appendix
\usepackage[toc,page]{appendix}
% title
\title{Personal notes on gravitational waves}
% author
\author{Jose Perdiguero Garate}

\begin{document}
\maketitle
\tableofcontents

\section*{Notation and conventions}

Throughout this notes I will be using the metric signature
\begin{equation}
    \left(-1, +1, +1, +1\right).
\end{equation}
Greek indices run over the four dimensional manifold spacetime, 
from $0$ to $3$, whereas latin indices run over only the spatial 
dimensions, from $1$ to $3$. Additionally, I am using the Einstein 
summation convention, where, repeated indices, indicate a summation.

The Minkowski metric tensor is written as $\eta_{\mu\nu}$ and 
an arbitrary metric tensor is $g_{\mu\nu}$. Partial derivatives 
are denoted by $\partial$ which can be acting on the four dimensions
or three dimensions, and can be written with a commas $g_{\mu\nu,\alpha}$. 
Covariant derivatives are defined with the symmetric standard connection
$\nabla$, which is compatible with the metric tensor such that 
$\nabla_{\alpha}g_{\beta\gamma} = g_{\beta\gamma;\alpha} = 0$
where I used semicolons for its denotation.

\section{Linear general relativity}

In this section I am presenting a general overview on how to build-up
the Einstein's field equation, given a background metric tensor and a 
perturbation tensor.

\subsection{General overview}

Decomposed the metric tensor as the sum of a background metric, in this case 
the flat spacetime Minkowski metric $\eta_{\mu\nu}$,plus a perturbation 
$h_{\mu\nu}$ as follows
\begin{equation}
    \label{metric tensor}
    g_{\mu\nu} =  \eta_{\mu\nu} + h_{\mu\nu}.
\end{equation}
Using the definition of the Kronecker delta object it is straightforward to
obtain the inverse tensor of the perturbation
\begin{equation}
    \label{definition delta}
    \delta^{\mu}_{\nu} = g^{\mu\sigma}g_{\sigma\nu}.
\end{equation}
Replacing Eq.\eqref{metric tensor} in to Eq.\eqref{definition delta}, and
neglecting second order terms in the perturbation field, leads to
\begin{equation}
    h^{\mu\nu} = -\eta^{\mu\alpha}\eta^{\nu\beta}h_{\alpha\beta},
\end{equation}
notice that, in order to upper/lower the indices of the perturbation I am only
using the background metric.\footnote{Including the perturbation tensor leads
to second order terms, which I am ignoring.} Schematically, second order terms
are neglected 
\begin{align}
    h_{\mu\nu}h_{\alpha\beta} & \sim 0, & 
    h_{\mu\nu}\partial_{\gamma} h_{\alpha\beta} & \sim 0, &
    \partial_{\delta} h_{\mu\nu}\partial_\gamma h_{\alpha\beta} & \sim 0.
\end{align}
This is the general overview of the fundamental field of general relativity,
which at its core is the metric tensor. In the following subsection, I will
be computing the Einstein's field equations for the metric tensor written 
in Eq.\eqref{metric tensor}.

\subsection{Geometrical objects}

The first object that is required to compute the Einstein's field equations is the
connection. Working on a torsion-free manifold, the Levi-Civita connection is written
as
\begin{equation}
    \label{levi civita connection}
    \Gamma^{\mu}{}_{\alpha\beta} = \frac{1}{2}g^{\mu\rho}\left(
        \partial_{\alpha}g_{\rho\beta} + \partial_{\beta}g_{\rho\alpha} -
        \partial_{\rho}g_{\alpha\beta}.
    \right)
\end{equation}
Replacing Eq.\eqref{metric tensor} in to Eq.\eqref{levi civita connection} leads to
\begin{equation}
    \Gamma^{\mu}{}_{\alpha\beta} = \frac{1}{2}\left(\eta^{\mu\rho} - h^{\mu\rho}\right)
        \left(\partial_{\alpha}\eta_{\rho\beta} 
        + \partial_{\beta}\eta_{\rho\alpha} 
        - \partial_{\rho}\eta_{\alpha\beta}
        + \partial_{\alpha}h_{\rho\beta} 
        + \partial_{\beta}h_{\rho\alpha} 
        - \partial_{\rho}h_{\alpha\beta}\right).
\end{equation}
Notice that, the only non-trivial contributions are the ones that are linear in the
perturbation, additionally, the partial derivatives of the Minkowski's metric tensor vanishes,
therefore, the connection coefficients are reduced to
\begin{equation}
    \label{levi civita connection flat + perturbation}
    \Gamma^{\mu}{}_{\alpha\beta} = \frac{1}{2}\eta^{\mu\rho}
        \left(\partial_{\alpha}h_{\rho\beta} + \partial_{\beta}h_{\rho\alpha} 
        - \partial_{\rho}h_{\alpha\beta}\right).
\end{equation}

Next, compute the Riemann curvature tensor
\begin{equation}
    \label{riemann curvature tensor}
    \mathcal{R}^{\rho}{}_{\sigma\mu\nu} = \partial_{\mu}\Gamma^{\rho}{}_{\sigma\nu}
    - \partial_{\nu}\Gamma^{\rho}{}_{\sigma\mu} 
    + \Gamma^{\gamma}{}_{\nu\sigma}\Gamma^{\rho}{}_{\mu\gamma}
    + \Gamma^{\gamma}{}_{\mu\sigma}\Gamma^{\rho}{}_{\nu\gamma},
\end{equation}
however, instead of computing directly from the above equation, it is convenient to 
notice the structure of the curvature tensor. The last two terms are quadratic in the 
Levi-Civita connection, and, since the connection is written with perturbation, then,
square terms in the connection vanishes, reducing the Riemann curvature tensor to
\begin{equation}
    \label{curvature tensor flat}
    \mathcal{R}^{\rho}{}_{\sigma\mu\nu} = \partial_{\mu}\Gamma^{\rho}{}_{\sigma\nu}
    - \partial_{\nu}\Gamma^{\rho}{}_{\sigma\mu}.
\end{equation}
Replacing Eq.\eqref{levi civita connection flat + perturbation} in to 
Eq.\eqref{curvature tensor flat} leads to
\begin{equation}
    \mathcal{R}^{\rho}{}_{\sigma\mu\nu} = \frac{1}{2}\eta^{\rho \alpha}\partial_{\mu}\left(
        \partial_{\sigma}h_{\alpha\nu} + \partial_{\nu}h_{\sigma\alpha}
        - \partial_{\alpha}h_{\sigma\nu}\right) +
        \frac{1}{2}\eta^{\rho \alpha}\partial_{\nu}\left(
        \partial_{\sigma}h_{\alpha\mu} + \partial_{\mu}h_{\sigma\alpha}
        - \partial_{\alpha}h_{\sigma\mu}\right),
\end{equation}
the above expression, can be simplified to
\begin{equation}
    \mathcal{R}^{\rho}{}_{\sigma\mu\nu} = \frac{1}{2}\eta^{\rho\alpha}\left(
        \partial_{\mu}\partial_{\sigma}h_{\alpha\nu} 
        - \partial_{\mu}\partial_{\alpha}h_{\sigma\nu}
        - \partial_{\nu}\partial_{\sigma}h_{\mu\alpha}
        + \partial_{\nu}\partial_{\alpha}h_{\mu\sigma}\right).
\end{equation}

From the Riemann tensor, it is straightforward to compute the Ricci tensor, by
contracting their respective indices
\begin{equation}
    \mathcal{R}_{\sigma\nu} = \mathcal{R}^{\mu}{}_{\sigma\mu\nu}.
\end{equation}
A direct computation shows the structure of the Ricci tensor
\begin{equation}
    \label{ricci tensor flat + perturbation}
    \mathcal{R}_{\sigma\nu} = \frac{1}{2}\left(
        \partial^{\alpha}\partial_{\sigma}h_{\alpha\nu} 
        -  \square h_{\sigma\nu} - \partial_{\nu}\partial_{\sigma}h
        + \partial_{\nu}\partial^{\alpha}h_{\alpha\sigma}\right),
\end{equation}
where $\square$ is the d'Alembert operator and $h$ is the trace of the perturbation.

In the same spirit, the curvature scalar can be obtained directly through
the contraction of the Ricci tensor
\begin{equation}
    \mathcal{R} = g^{\mu\sigma}\mathcal{R}_{\mu\sigma}.
\end{equation}
This computation is straightforward
\begin{equation}
    \mathcal{R} = \left(\eta^{\mu\sigma} - h^{\mu\sigma}\right) \frac{1}{2}\left(
        \partial^{\alpha}\partial_{\sigma}h_{\alpha\nu} 
        -  \square h_{\sigma\nu} - \partial_{\nu}\partial_{\sigma}h
        + \partial_{\nu}\partial^{\alpha}h_{\alpha\sigma}\right).
\end{equation}
Neglecting second order terms in the perturbation field, the scalar curvature is given
by
\begin{equation}
    \label{curvature scalar flat + perturbation}
    \mathcal{R} = \partial_{\mu}\partial_{\sigma}h^{\mu\sigma} - \square h
\end{equation}

Now, we can compute the Einstein's field equations without a cosmological constant
\begin{equation}
    \label{Einsteins field equation}
    \mathcal{R}_{\mu\nu} - \frac{1}{2}\mathcal{R}g_{\mu\nu} = \frac{8\pi G}{c^4}
    \mathcal{T}_{\mu\nu}
\end{equation}
where $\mathcal{T}_{\mu\nu}$ is the energy momentum tensor. Replacing 
Eq.\eqref{ricci tensor flat + perturbation} and Eq.\eqref{curvature scalar flat + perturbation}
in to Eq.\eqref{Einsteins field equation} leads to
\begin{equation}
        \partial^{\alpha}\partial_{\mu}h_{\alpha\nu} 
        -  \square h_{\mu\nu} - \partial_{\nu}\partial_{\mu}h
        + \partial_{\nu}\partial^{\alpha}h_{\alpha\mu}
        - \left(\partial_{\alpha}\partial_{\beta}h^{\alpha\beta} - \square h\right)
        \left(\eta_{\mu\nu} + h_{\mu\nu}\right) = \frac{16\pi G}{c^4}\mathcal{T}_{\mu\nu}.
\end{equation}
Just like before, neglecting second order terms in the perturbation, the above equation
can be reduced to
\begin{equation}
    \label{einsteins equation flat + perturbation}
        \partial^{\alpha}\partial_{\mu}h_{\alpha\nu} 
        -  \square h_{\mu\nu} - \partial_{\nu}\partial_{\mu}h
        + \partial_{\nu}\partial^{\alpha}h_{\alpha\mu}
        - \eta_{\mu\nu}\partial_{\alpha}\partial_{\beta}h^{\alpha\beta} 
        - \eta_{\mu\nu}\square h = \frac{16\pi G}{c^4}\mathcal{T}_{\mu\nu}.
\end{equation}
The above equation, can be written in a much more compact manner by using the following
variable change\footnote{In the standard literature $X_{\mu\nu}$ is written as 
$\bar{h}_{\mu\nu}$, but I strongly believed this leads to confusions.}
\begin{equation}
    X_{\mu\nu} = h_{\mu\nu} - \frac{1}{2}\eta_{\mu\nu}h,
\end{equation}
which can be inverted through standard methods
\begin{equation}
    \label{variable change}
    h_{\mu\nu} = X_{\mu\nu} - \frac{1}{2}\eta_{\mu\nu}X
\end{equation}
where $X$ is the trace of the tensor $X_{\mu\nu}$, and the tensor $X_{\mu\nu}$, also satisfies
the relation $X = -h$. Replacing the variable change written in Eq.\eqref{variable change} in 
Eq.\eqref{einsteins equation flat + perturbation} and simplifying terms, leads to
\begin{equation}
    \label{efe X}
    \partial^{\alpha}\partial_{\mu}X_{\alpha\nu} + \partial_{\nu}\partial^{\alpha}X_{\alpha\mu}
    - \square X_{\mu\nu} - \eta_{\mu\nu}\partial_{\alpha}\partial_{\beta}X^{\alpha\beta}
    =  \frac{16\Pi G}{c^4}\mathcal{T}_{\mu\nu},
\end{equation}
which, in some sense has a more simple structure that Eq.\eqref{einsteins equation flat + perturbation},
and also contains the wave operator. Nonetheless, this does not look like a gravitational 
wave equations. In the next subsection, I will show you, how can you derived the gravitational
wave equation from the above expression using a gauge transformation.

\subsection{Gauge transformation}

Consider the infinitesimal gauge coordinate transformation
\begin{equation}
    \label{gauge transformation}
    x^{\mu} \longrightarrow x'^{\mu} =  x^{\mu} + \xi^{\mu},
\end{equation}
where $\xi^{\mu}$ is a small vector. Then, it is possible to obtain the relation
between and the inverse relation of the coordinate transformation of their respective
derivatives 
\begin{align}
    \label{gauge transformation derivatives}
    \frac{\partial x'^{\alpha}}{\partial x^{\beta}} & = \delta^{\alpha}_{\beta} 
    + \partial_{\beta} \xi^{\alpha} & 
    \frac{\partial x^{\alpha}}{\partial x'^{\beta}} & = \delta^{\alpha}_{\beta} 
    - \partial_{\beta} \xi^{\alpha}.
\end{align}

Using the above information, the metric tensor under a gauge coordinate transformation
changes as 
\begin{equation}
    g'_{\alpha\beta} = \frac{\partial x^{\mu}}{\partial x'^{\alpha}}
    \frac{\partial x^{\nu}}{\partial x'^{\beta}}g_{\mu\nu}.
\end{equation}
Using Eq.\eqref{gauge transformation derivatives} in the above equation, and neglecting
second order terms in the perturbation, leads to
\begin{equation}
    g'_{\alpha\beta} = g_{\alpha\beta} - g_{\alpha\nu}\partial_{\beta}\xi^{\nu}
    - g_{\mu\beta}\partial_{\alpha}\xi^{\mu}
\end{equation}
replacing the expressions for the metric tensor, see Eq.\eqref{metric tensor}, leads to
\begin{equation}
    \label{gauge transformation of perturbation}
    h'_{\alpha\beta} = h_{\alpha\beta} - \partial_{\beta}\xi_{\alpha} 
    - \partial_{\alpha}\xi_{\beta}.
\end{equation}
Is worth mention that the Riemann tensor is invariant under a gauge transformation. 
Therefore, we have the freedom to choose or fix the vector $\xi^{\mu}$ as we liked. 
Additionally, Eq.\eqref{gauge transformation of perturbation} is only valid using a 
Minkowski background, if we were working on a curved spacetime background, there will
be additional terms, the most general expression can be found in at the end of this 
section. As the rule of gauge transformation is written in 
Eq.\eqref{gauge transformation of perturbation}, then is trivial to compute the gauge
transformation of the auxiliary variable $X_{\mu\nu}$
\begin{equation}
    X_{\mu\nu} \to X'_{\mu\nu} = h'_{\mu\nu} 
    - \frac{1}{2}\eta_{\mu\nu}\eta^{\alpha\beta}h'_{\alpha\beta},
\end{equation}
replacing the transformation rule leads to
\begin{equation}
    X'_{\mu\nu} = h_{\mu\nu} - \partial_{\nu}\xi_{\mu} 
    - \partial_{\mu}\xi_{\nu}
    - \frac{1}{2}\eta_{\mu\nu}\eta^{\alpha\beta}\left(
    h_{\alpha\beta} - \partial_{\beta}\xi_{\alpha} - \partial_{\alpha}\xi_{\beta}\right).
\end{equation}
The above expression can be simplified 
\begin{equation}
    \label{gauge transformation X}
    X'_{\mu\nu} = X_{\mu\nu} - \partial_{\mu}\xi_{\nu} 
    - \partial_{\nu}\xi_{\mu} + \eta_{\mu\nu}\partial^{\sigma}\xi_{\sigma},
\end{equation}
where I used the definition of the $X_{\mu\nu}$ tensor written in Eq.\eqref{variable change}. 
Now, it is convenient to work with upper index, thus, using the inverse Minkowski metric we 
can upper every index
\begin{equation}
    X'^{\alpha\beta} = X^{\alpha\beta} - \eta^{\mu\alpha}\partial_{\mu}\xi^{\beta} 
    - \eta^{\nu\beta}\partial_{\nu}\xi^{\alpha} 
    + \eta^{\alpha\beta}\partial^{\sigma}\xi_{\sigma}.
\end{equation}
At this point we take the divergence of $X_{\mu\nu}$
\begin{equation}
    \partial_{\beta}X'^{\alpha\beta} = \partial_{\beta}X^{\alpha\beta}
    - \eta^{\mu\alpha}\partial_{\beta}\partial_{\mu}\xi^{\beta} 
    - \eta^{\nu\beta}\partial_{\beta}\partial_{\nu}\xi^{\alpha} 
    + \eta^{\alpha\beta}\partial_{\beta}\partial^{\sigma}\xi_{\sigma},
\end{equation}
after simplification of terms, leads to
\begin{equation}
    \partial_{\beta}X'^{\alpha\beta} = \partial_{\beta}X^{\alpha\beta}
    - \square \xi^{\alpha}.
\end{equation}
At this point, recall that we still have the freedom to choose the $\xi^{\mu}$. Therefore,
fixing 
\begin{equation}
    \label{gauge condition}
    \square \xi^{\alpha} = \partial_{\beta}X^{\alpha\beta} 
    \rightarrow \partial_{\beta}X'^{\alpha\beta} = 0,
\end{equation}
this is known as the Lorentz gauge, which vanishes the vast majorities of 
terms of Eq.\eqref{efe X}. The only, non-trivial term comes from the wave 
operator, leading to
\begin{equation}
    \square X_{\mu\nu} = -\frac{16\Pi G}{c^4}\mathcal{T}_{\mu\nu},
\end{equation}
which for the special vacuum case $\mathcal{T}_{\mu\nu} = 0$, reduces the 
above equation to 
\begin{equation}
    \square X_{\mu\nu} = 0.
\end{equation}
The above result, is known as the gravitational wave equation. In the next subsection we will
be dealing with the problem of counting the degrees of freedom of a gravitational wave.

\subsection{Degrees of freedom}

The perturbation has 16 independent components which must be determined. 
However, because it is defined as a symmetric tensor it only has 10 
independent components. Moreover, due to the gauge condition 
Eq.\eqref{gauge condition} there are four additional equations/constraint 
that must be satisfied, leaving only 6 independent components. Nonetheless, 
this choice, does not completely fixes the gauge, we still have the freedom 
to choose to fix the four components of the displacement vector $\xi^{\mu}$, 
and hence, reduced even further the number of independent components of the 
perturbation to 2 independent components. This is known as the residual gauge.

An intuitive way of seen this reduction of the independent components of the
perturbation tensor comes from knowing that initially there are 10 independent
components of a symmetric tensor of rank $2$ tensor. Then by using the Lorentz
gauge restriction 
\begin{equation}
    \partial_{\beta}X'^{\alpha\beta} = 0,
\end{equation}
which lead to four different equations, meaning that, there are four more 
constraint to impose, reducing the number of 10 independent components to
6. Then, we can perform another gauge transformation by an infinitesimal
vector displacement $\xi^{\mu}$ 
\begin{equation}
    X'_{\mu\nu} \rightarrow X'_{\mu\nu} + \xi_{\mu\nu},
\end{equation}
where $\xi_{\mu\nu}$ is defined as 
\begin{equation}
    \xi_{\mu\nu} \equiv \eta_{\mu\nu}\partial^{\alpha}\xi_{\alpha} 
    -\partial_{\mu}\xi_{\nu} -\partial_{\nu}\xi_{\mu}.
\end{equation}
Therefore, choosing properly the vector $\xi_{\mu}$ such that $\square \xi_{\mu\nu} = 0$,
it is possible to reduced 4 more degrees of freedom of the perturbation 
tensor. Leaving only $2$ independent components, which are known as the $2$
polarization, $+$ and $x$.

Exploiting the residual gauge symmetry, it is possible to eliminate
directly components of the perturbation tensor. Using Eq.\eqref{gauge transformation X}
it is possible to vanishes the trace and the spatial-temporal components of the 
perturbation $X = 0$ and $X_{0i} = 0$. As a consequence of this, the perturbation
and the variable change are the same $X_{\mu\nu} = h_{\mu\nu}$. From the Lorentz gauge
condition Eq.\eqref{gauge condition}
\begin{equation}
    \dot{h}_{00} + \partial_{i}h_{i0} = 0,
\end{equation}
from which, we infer that the temporal-temporal components is a function of only
the spatial coordinates. As this component does not depend on the time coordinate, (
which is our concern, because gravitational waves are time-dependent), we can set 
$h_{00} = 0$. In a nutshell, the constraint are
\begin{align}
    h_{\mu 0 } & = 0, & h & = 0, & \partial_{i}h_{ij} & = 0.
\end{align}
This is known as the \textit{transverse-traceless} gauge (TT).


\subsection{Scalar-Vector-Tensor decomposition}

Although we have found so far the gravitational wave equation coupled with a
energy-momentum tensor, in principle we are done. Now, the next task should be
to compute the Einstein's field equation for a given symmetry of the metric tensor.
However, we can take advantage of the \textit{scalar-vector-tensor} decomposition
to write down the independent components of Einstein's equations in a much more 
simple form.

First, lets decomposed the perturbation tensor as follows
\begin{align}
    h_{00} & = -2\phi, \\
    h_{0i} & = \partial_i B + S_i, \\
    h_{ij} & = h_{ij} + \partial_{i} F_{j} + \partial_{j} F_{i} - 2\psi\delta_{ij}, 
    + \left(\partial_{i}\partial_{j} - \frac{1}{3}\delta_{ij}\nabla^{2}\right)E,
\end{align}
where $\phi$, $B$, $\psi$ and $E$ are scalars functions, $S_i$ and $F_i$ 
are vectors whose divergence vanishes $\partial_{i}S_{i} = \partial_{i}F_{i} = 0$,
and $h_{ij}$ is a tensor whose divergence is trivial $\partial_{i}h_{ij} = 0$ and also
by construction must be traceless $h = 0$. Notice that, in principle, the perturbation
tensor has $10$ independent components, and from the decomposition, we have $4$ scalars,
$6$ components for the two vectors $-2$ constraints, and from the tensorial part
there are only $6$ independent components $-4$ constraints. Therefore, the decomposition
leads to $4 + \left(6 - 2\right) + \left(6 - 4\right) = 10$, which, of course, matches
the number of independent components of the perturbation. Details on how to derived this
decomposition can be found in Ref.\cite{poisson2014gravity} in section $5.5$.

In the same way, the vector displacement $\xi_{\mu}$ can also be decomposed
\begin{equation}
    \xi_{\mu} = \left(d_0, \partial_i d + d_i\right),
\end{equation}
and using the gauge transformation rule written in Eq.\eqref{gauge transformation of perturbation},
it is straightforward to derived how the components of the perturbation tensor transform
under a coordinate gauge transformation. Take for example the $h_{tt}$ component
\begin{equation}
    h'_{tt} = h_{tt} - 2\partial_{t}\xi_{t},
\end{equation}
which, replacing the values leads to
\begin{equation}
    2\phi' = 2\phi + 2\dot{d}_0,
\end{equation}
then, the rule of transformation can be written as
\begin{equation}
    \phi' = \phi + \dot{d}_0.
\end{equation}
The same idea should be applied to the different components of the SVT decomposition. The scalar
components transform as follow
\begin{align}
    \phi' & = \phi + \dot{d}_0, & B' & = B - d_0 - \dot{d} \\
    \psi' & = \psi +\frac{1}{3}\nabla^2d & E & = E - 2d,
\end{align}
the vector parts transfrom as
\begin{align}
    S_i' & = S_i - \dot{d}_i, & F_i & = F_i - d_i.
\end{align}
and finally the tensor part
\begin{equation}
    h'_{ij} = h_{ij}
\end{equation}
Once the rules of gauge transformation are known for the STV decomposition, it is
straightforward to build-up gauge invariants using the components of the perturbation, 
these gauge invariants are
\begin{align}
    \Phi & \equiv \phi + \dot{B} - \frac{1}{2}\ddot{E}, &
    \Theta & \equiv -2\psi - \frac{1}{3}\nabla^2E, &
    \Sigma_{i} & \equiv S_i - \dot{F}_i, &
    h_{ij} & \equiv h_{ij},
\end{align}
there are two scalar gauge invariants, one vector invariant and one tensor invariant
quantity. It is straightforward the proof that these are gauge invariants, lets 
consider the first gauge invariant object and perform the coordinate gauge transformation
\begin{equation}
    \Phi \to \Phi' = \phi' + \dot{B}' - \frac{1}{2}\ddot{E}',
\end{equation}
replacing the rules of transformation for each object leads to
\begin{equation}
    \Phi' = \Phi + \dot{d}_0 + \dot{B} - \dot{d_0} - \ddot{d} - 
    \frac{1}{2}\left(\ddot{E} - 2\ddot{d}\right),
\end{equation}
which, after simple algebra simplification turns to
\begin{equation}
    \Phi' = \Phi + \dot{B} - \frac{1}{2}\ddot{E}.
\end{equation}
The proof for the other three gauge invariants follows the same idea.

Lets compute the geometrical part of Einstein's field Eq. 
\eqref{einsteins equation flat + perturbation}, to make this more easy to see, I will write 
down the equation
\begin{equation}
        \partial^{\alpha}\partial_{\mu}h_{\alpha\nu} 
        -  \square h_{\mu\nu} - \partial_{\nu}\partial_{\mu}h
        + \partial_{\nu}\partial^{\alpha}h_{\alpha\mu}
        - \eta_{\mu\nu}\partial_{\alpha}\partial_{\beta}h^{\alpha\beta} 
        - \eta_{\mu\nu}\square h = \frac{16\pi G}{c^4}\mathcal{T}_{\mu\nu}.
\end{equation}
Let's compute the $tt$ components of the LHS of the above equation. A straightforward
computation leads to
\begin{align}
    & \left(2\ddot{\phi} + \nabla^{2}\dot{B}\right) + \left(-2\ddot{\phi} + 2\nabla^2\phi\right)
    + \left(6\ddot{\psi} + 2\ddot{\phi}\right) + \left(2\ddot{\phi} + \nabla^2\dot{B}\right)\\
    & + \left(-2\ddot{\phi} - 2\nabla^2 \dot{B} -2\nabla^2\psi - \frac{2}{3}\nabla^2\nabla^2E\right)
    +\left(-6\ddot{\psi} + 6\nabla^2\psi -2\ddot{\phi} - 2\nabla^2\phi\right), 
\end{align}
which after simplification is reduced to
\begin{equation}
    G_{00} = \frac{1}{3}\nabla^2\left(-2\psi - \nabla^2 E\right) \to
    G_{00} = \nabla^2\Theta,
\end{equation}
alternatively, it can be written in terms of gauge invariant as seen in the above right side. in
a similar manner, it can be computed out all the other components of the Einstein tensor
\begin{align}
    G_{00} & = -\nabla^2\theta, \\
    G_{0i} & = -\frac{1}{2}\nabla^2\Sigma_i -\partial_{i}\dot{\Theta},\\
    G_{ij} & = -\frac{1}{2}\square h_{ij} - \partial_{(i}\dot{\Sigma}_{j)} - \frac{1}{2}\partial_{i}
    \partial_{j}\left(2\Phi + \Theta\right) + \delta_{ij}\left(\frac{1}{2}\nabla^2\left(2\Phi + \Theta\right)
    - \ddot{\Theta} \right)
\end{align}

A similar procedure can be applied to the stress energy momentum tensor
\begin{align}
    \label{energy momentum tensor svt}
    \mathcal{T}_{00} & = \rho, \\ 
    \mathcal{T}_{0i} & = \partial_{i}u + u_i, \\
    \mathcal{T}_{ij} & = \Pi_{ij} + \partial_{i}v_{j} + \partial_{j}v_{i} + p\delta_{ij}
    + \left(\partial_{i}\partial_{j} - \frac{1}{3}\delta_{ij}\nabla^{2}\right)\sigma,
\end{align}
where $\rho$, $u$, $p$ and $\sigma$ are scalar functions, $u_i$ and $v_i$ are vectors
whose divergence vanishes $\partial_{i}u_{i} = 0 $ and $\partial_{i}v_{i} = 0$, and $\Pi_{ij}$
is a traceless tensor (by construction), therefore $\Pi = 0$, whose divergence vanishes 
$\partial_{i}\Pi_{ij} = 0$. The natural next step, would be to build-up the rules of
gauge transformation of the components of the energy-momentum tensor, however because 
the energy-momentum tensor must be zero in the background, all the components
(perturbed) are gauge invariants, this is known as the Steward Walker lemma 
\cite{stewart1993advanced}.

From the conservation law, 
\begin{equation}
    \nabla_{\mu}\mathcal{T}^{\mu\nu} = \partial_{\mu}\mathcal{T}^{\mu\nu}
    + \Gamma^{\mu}{}_{\mu\lambda}\mathcal{T}^{\lambda\nu}
    + \Gamma^{\nu}{}_{\mu\lambda}\mathcal{T}^{\mu\lambda} \sim \partial_{\mu}\mathcal{T}^{\mu\nu}.
\end{equation}
Notice that the energy-momentum tensor is defined as a perturbation and the background
energy momentum tensor vanishes, therefore the second and third term can be neglected, leaving
only
\begin{equation}
    \partial^{t}\mathcal{T}_{t\nu} + \partial^{i}\mathcal{T}_{i\nu} = 0
\end{equation}. 
Working with the $t$ component leads to
\begin{equation}
    \partial^{t}\mathcal{T}_{tt} + \partial^{i}\mathcal{T}_{it} = 0,
\end{equation}
Replacing Eq.\eqref{energy momentum tensor svt} in the above equation leads to
\begin{equation}
    -\dot{\rho} + \partial^{i}\left(\partial_{i}u + u _ i\right) = 0,
\end{equation}
however, the last term vanishes out because its divergence is zero. Therefore, the above equation
is simplified to
\begin{equation}
    \nabla^2 u = \dot{\rho}.
\end{equation}
To obtain the other continuity equations from the conservation of the energy-momentum tensor follows
a similar idea. Replacing Eq.\eqref{energy momentum tensor svt} into the above equation leads to the following
constraints
\begin{align}
    \nabla^{2}u & = \dot{\rho}, & 
    \nabla^{2}\sigma & = \frac{3}{2}\left(\dot{u} - p\right), & 
    \nabla^{2}v_{i} & = \dot{u}_{i}, & 
\end{align}

The next step, is to put all this ingredients together into Einstein's field equation. A 
straightforward computation leads to the following field equations
\begin{align}
    \label{efe linear with matter}
    \nabla^2 \Theta & = -\rho, \\
    \nabla^2 \Phi & = \left(\rho + 3p - 3\dot{u}\right), \\
    \nabla^2 \Sigma_{i} & = - 2S_{i}, \\
    \square h_{ij} & = -2\Pi_{ij}.
\end{align}
These are the set of differential equations, where there is only one wave equations coming
from the $h_{ij}$. The other fields obey a Poisson-like equation. Notice that, both sides of 
the Eqs.\eqref{efe linear with matter} are gauge invariants, as they should be.

\subsection{Gravitational waves in a curved spacetime}

Now, lets move on one step a head and use the FRW metric as the background metric
\begin{equation}
    g_{\mu\nu} = \bar{g}_{\mu\nu} + h_{\mu\nu},
\end{equation}
where the bar quantities make references to background objects. The distance between two
points can be compute using
\begin{align}
    \mathrm{d}s^{2} = a^{2}(\eta)\left(-\mathrm{d}\eta^2 
    + \delta_{ij}\mathrm{d}x^{i}\mathrm{d}x^{j}
    + h_{ij}\mathrm{d}x^{i}\mathrm{d}x^{j}\right),
\end{align}
where $h_{ij}$ is in the TT-gauge, and I am also working in conformal time, defined 
by the relation 
\begin{equation}
    \label{conformal time relation}
    a\mathrm{d}\eta = \mathrm{d}t.
\end{equation} 
Next, you can compute the Christoffel symbols, the Riemann curvature tensor and the 
Ricci scalar. These are the geometrical objects that you need to compute the Einstein's 
field equations. The perturbed Einstein's field equation leads to just one equation
\begin{equation}
    h''_{ij} + 2\mathcal{H}h'_{ij} - \nabla^{2}h_{ij} = 8\pi G a^{2}\bar{p}\Pi_{ij},
\end{equation}
where the time derivatives are take with respect to the conformal time, and the Hubble function
$\mathcal{H}$ is defined also in conformal time. Using Eq.\eqref{conformal time relation} you can
work the above equation into physics time
\begin{equation}
    \ddot{h}_{ij} + 3H\dot{h}_{ij} - \frac{\nabla^{2}}{a^{2}}h_{ij} = 8\pi G \bar{p}\Pi_{ij}.
\end{equation}
From the above expression it is not obvious the classification type of solutions. However,
by using the following variable change
\begin{equation}
    X_{ij}\left(\bold{k},\eta\right) = a h_{ij}\left(\bold{k},\eta\right),
\end{equation}
the differential equation turns to 
\begin{equation}
    X_{ij}\left(\bold{k},\eta\right) + \left(k^2 - \frac{a''}{a}\right)X_{ij}\left(\bold{k},\eta\right)
    = 16\pi G a^3\Pi_{ij}.
\end{equation}
Looking for analytical solutions in vacuum $\Pi_{ij} = 0$, there is a clear distinction between
two branches of solutions. The first case is given by the condition $k^2 \gg \mathcal{H}^2$ known
as the sub-Hubble sphere
\begin{equation}
    X''\left(\bold{k},\eta\right) + k^2X\left(\bold{k},\eta\right) = 0,
\end{equation}
which is the equation of the well known harmonic oscillator, whose solution is
\begin{equation}
    X\left(\bold{k},\eta\right) = A_r(\bold{k})e^{ik\eta} + B_r(\bold{k})e^{-ik\eta},
\end{equation}
or in terms of the original variable
\begin{equation}
    h_{r}\left(\bold{k},\eta\right) = \frac{A_r}{a(\eta)}(\bold{k})e^{ik\eta} + 
    \frac{B_r}{a(\eta)}(\bold{k})e^{-ik\eta}.
\end{equation}
The second case is $k^2 \ll \mathcal{H}^2$, this are the super-Hubble spheres
\begin{equation}
    X''\left(\bold{k},\eta\right) - \frac{a''}{a}\left(\bold{k},\eta\right) = 0,
\end{equation}
whose solution in terms of the original variable is
\begin{equation}
    h_{r}\left(\bold{k},\eta\right) = A_r(\bold{k}) + 
    B_r(\bold{k})\int \frac{\mathrm{d}\eta}{a^2(\eta')}.
\end{equation}

To finalized this section, I just want to deduced the most general gauge transformation
of a rank 2 tensor, having a curved background. This is a more general expression that the one
found in Eq.\eqref{gauge transformation of perturbation}. To obtain this expression consider
the coordinate transformation
\begin{equation}
    B'_{\mu\nu}(x') = \frac{\partial x^\alpha}{\partial x^{\mu}}\frac{\partial x^\beta}{\partial x^{\nu}}
    B_{\alpha\beta}(x),
\end{equation}
under a gauge transformation it follows that
\begin{equation}
    B'_{\mu\nu}(x') = \left(\delta ^{\alpha}_{\mu} - \partial_{\mu}\xi^{\alpha}\right)
    \left(\delta ^{\beta}_{\nu} - \partial_{\nu}\xi^{\beta}\right)
    B_{\alpha\beta}(x),
\end{equation}
expanding the above expression and neglecting $\xi^2$ terms leads to
\begin{equation}
    B'_{\mu\nu}(x') = B_{\mu\nu}(x) -  B_{\mu\alpha}(x)\partial_{\nu}\xi^{\alpha}
    - B_{\beta\nu}(x)\partial_{\mu}\xi^{\beta}.
\end{equation}
Notice that, the LHS is in the $x'$ coordinate system, whereas the RHS is written in $x$
coordinate system. In order solve the discrepancy, it is required to perform a Taylor series
expansion
\begin{align}
    B'_{\mu\nu}(x') & = B_{\mu\nu}(x' - \xi) -  B_{\mu\alpha}(x' - \xi)\partial_{\nu}\xi^{\alpha}
    - B_{\beta\nu}(x' - \xi)\partial_{\mu}\xi^{\beta},\\
    B'_{\mu\nu}(x') & = B_{\mu\nu}(x') - \xi^{\alpha}\partial_{\alpha}B_{\mu\nu}(x') -  
    B_{\mu\alpha}(x')\partial_{\nu}\xi^{\alpha} - B_{\beta\nu}(x')\partial_{\mu}\xi^{\beta}.
\end{align}
At this point, the $B$ tensor can be decomposed with as a background plus a perturbation, and 
considering that the background in both coordinate system are similar, leads to
\begin{equation}
    \label{general gauge transformation}
    \delta B'_{\mu\nu} = \delta B_{\mu\nu} - \bar{B}_{\mu\alpha}\partial_{\nu}\xi^{\alpha}
    - \bar{B}_{\beta\nu}\partial_{\mu}\xi^{\beta} - \xi^{\alpha}\partial_{\alpha}\bar{B}_{\mu\nu}.
\end{equation} 
Notice that, if the background quantity is trivial (vanishes), then, all the perturbation are by 
construction gauge invariants. This is a direct proof of the Steward-Walker lemma. Additionally,
Eq. \eqref{general gauge transformation} was derived for an arbitrary background, if we use this
result to compute how the perturbation $h_{\mu\nu}$ transform leads to
\begin{align}
    h'_{\mu\nu} = h_{\mu\nu} - h_{\mu\alpha}\partial_{\nu}\xi^{\alpha}
    - h_{\beta\nu}\partial_{\mu}\xi^{\beta} - \xi^{\alpha}\partial_{\alpha}h_{\mu\nu},
\end{align}
however, the last term vanishes due to its higher order in the perturbation series. Therefore,
the above expression is reduced to one already found in 
Eq. \eqref{gauge transformation of perturbation}.


\section{Chern-Simons gravity}

Consider the Chern-Simons action coupled with the Einstein-Hilbert action is
\begin{equation}
    S = \int \mathrm{d}^4x \sqrt{-g}\mathcal{R}
\end{equation}

\section{Polynomial parity breaking}

The most general gravitational wave propagation equation is of the form
\begin{equation}
    Ah'' + Bh' + Ch' = 0,
\end{equation}
where time derivatives are taking in conformal time. Without any loss of generalities 
and assuming $A \neq 0 $ the above equation can be written as 
\begin{equation}
    h'' + \bar{B}h' + \bar{C}h = 0.
\end{equation}
The idea is to write down the most generality gravitational wave propagation equation that 
breaks the symmetry of parity, and the fundamental building blocks we can use are 
\begin{equation}
     k ,\mathcal{H} ,\Lambda a ,
\end{equation}
which are the only elements that have units of inverse of conformal time, this is a necessary
constraint that must be satisfied. Writing down the most general polynomial expression using those
fields are
\begin{align}
    \bar{B} & = k\bar{B}_{k} + \mathcal{H}\bar{B}_{\mathcal{H}} + \Lambda a\bar{B}_{\Lambda}
    + f(\varphi')\bar{B}_{\varphi}, \label{B polynomial}\\
    \bar{C} & = k\bar{C}_{kk} + \mathcal{H}\bar{C}_{\mathcal{H}\mathcal{H}} 
    + \Lambda a\bar{C}_{\Lambda\Lambda} + f(\varphi')\bar{C}_{\varphi\varphi}. \label{C polynomial}
\end{align}
It is possible to write down the coefficients as a dimensionless quantities 
\begin{align}
    \bar{B}_{i} = \sum b^{i}_{nm}\left(\eta\right)\left(\frac{k}{\Lambda a}\right)^{n}
    \left(\frac{\mathcal{H}}{\Lambda a}\right)^{m}, \label{B sum} \\
    \bar{C}_{ij} = \sum c^{ij}_{nm}\left(\eta\right)\left(\frac{k}{\Lambda a}\right)^{n}
    \left(\frac{\mathcal{H}}{\Lambda a}\right)^{m} \label{C sum}, 
\end{align}
where $n>0$ and $m>0$, additionally, for consistency, the polynomial functions must be well defined
in the limits, such that it is possible to recover the Minkowski spacetime
\begin{align}
    \mathcal{H} &\to 0, & k &\to 0, & f(\varphi') & \to 0
\end{align}
Notice that, in Eq. \eqref{B polynomial} there are three fundamental building blocks, however, in 
its respective equation Eq. \eqref{B sum} there are only two fundamental fields, the reason for that,
is the degeneracy of the fields. It is possible to compute all configurations of those three fields
using only those two written down in Eq. \eqref{B sum}, to see this, consider the following example:
lets compute one coefficient 
\begin{equation}
    n = 1, m = 0 \to k\bar{B}_{k} = k b^{k}_{1,0}\left(\frac{k}{\Lambda a}\right) 
    = b^{k}_{1,0}\frac{k^2}{\Lambda a}
\end{equation}
and compute the degenerate term that produces the same contribution
\begin{equation}
    n = 2, m = 0 \to \Lambda a \bar{B}_{\Lambda} = \Lambda a b^{\Lambda}_{2,0}\left(\frac{k}{\Lambda a}\right)^2
    = b^{\Lambda}_{2,0} \frac{k^2}{\Lambda a},
\end{equation}
as you can see, the term $b^{\Lambda}_{20}$ is already contained in $b^{\kappa}_{10}$. Lets work 
another example
\begin{equation}
    n = 0, m = 1 \to k\bar{B}_{k} = k b^{k}_{0,1}\left(\frac{\mathcal{H}}{\Lambda a}\right) 
    = b^{k}_{0,1}\frac{k\mathcal{H}}{\Lambda a},
\end{equation}
and the degenerate term that leads to the same result
\begin{equation}
    n = 1, m = 0 \to \mathcal{H}\bar{B}_{\mathcal{H}} = \mathcal{H} b^{\mathcal{H}}_{1,0}\left(\frac{k}{\Lambda a}\right)
    = b^{k}_{1,0}\frac{k\mathcal{H}}{\Lambda a},
\end{equation}
as expected, $b^{k}_{10} = b^{k}_{01}$. Lets work one final example,
\begin{equation}
    n = 3, m = 0 \to k\bar{B}_{k} = k b^{k}_{3,0}\left(\frac{k}{\Lambda a}\right)^3 
    = b^{k}_{3,0}\frac{k^4}{\left(\Lambda a\right)^3},
\end{equation}
and the other term
\begin{equation}
    n = 4, m = 0 \to \Lambda a\bar{B}_{\Lambda} = \Lambda a b^{k}_{4,0}\left(\frac{k}{\Lambda a}\right)^4 
    = b^{k}_{4,0}\frac{k^4}{\left(\Lambda a\right)^3}.
\end{equation}
This should make clear, that, the only two fundamental building blocks necessary to define the most
general polynomial term is
\begin{equation}
    \label{B building blocks}
    \bar{B} = \mathcal{H}\bar{B}_{\mathcal{H}} + \Lambda a \bar{B}_{\Lambda},
\end{equation}
and also neglecting second order terms in $\mathcal{H}$. A similar analysis can be done for the 
$\bar{C}$ contribution, leading two only two types of term
\begin{equation}
    \label{C building blocks}
    \bar{C} = k^2 \bar{C}_{kk} + \left(\Lambda a\right)^2 \bar{C}_{\Lambda\Lambda}.
\end{equation}

Assuming small deviations from general relativity due to the parity breaking symmetry, we can expand
the coefficients as follows
\begin{align}
    b^{i}_{n,m} = \bar{b}^{i}_{n,m} + \lambda^{n}_{r,l}\lambda^{m}_{r,l} \delta b^{i}_{n,m} \label{B deviation},\\
    c^{ij}_{n,m} = \bar{c}^{ij}_{n,m} + \lambda^{n}_{r,l}\lambda^{m}_{r,l} \delta c^{ij}_{n,m} \label{C deviation}.
\end{align}
Replacing the above expression for $b^{i}_{n,m}$ in Eq. \eqref{B building blocks} leads to
\begin{align}
    \label{gw equation pv}
    \bar{B} & = \sum_{n,m} \mathcal{H}\left(\bar{b}^{\mathcal{H}}_{n,m} 
    + \lambda^{n}_{r,l}\lambda^{m}_{r,l} \delta b^{\mathcal{H}}_{n,m}\right)
    \left(\frac{k}{\Lambda a}\right)^{n}\left(\frac{\mathcal{H}}{\Lambda a}\right)^{m} \notag \\
    & + \sum_{n,m} \Lambda a \left(\bar{b}^{\Lambda}_{n,m} 
    + \lambda^{n}_{r,l}\lambda^{m}_{r,l} \delta b^{\Lambda}_{n,m}\right)
    \left(\frac{k}{\Lambda a}\right)^{n}\left(\frac{\mathcal{H}}{\Lambda a}\right)^{m}.
\end{align}
It is possible to extract more information from the above expression by considering the general
relativity background. Consider the case $n = m = 0$ then
\begin{equation}
    \bar{B} = \mathcal{H}\bar{b}^{\mathcal{H}}_{0,0} + \mathcal{H} 
    \lambda^{n}_{r,l}\lambda^{m}_{r,l} \delta b^{\mathcal{H}}_{0,0} +
    \Lambda a \bar{b}^{\Lambda}_{0,0} 
    + \lambda^{n}_{r,l}\lambda^{m}_{r,l} \delta \bar{b}^{\Lambda}_{0,0}, 
\end{equation}
from which is clear that, in order to recover general relativity it is required that (this is only
by working with the background)
\begin{align}
    \bar{b}^{\mathcal{H}}_{0,0} & = 2, & \bar{b}^{\mathcal{H}}_{n,m} & = 0, & 
    \bar{b}^{\Lambda}_{n,m} & = 0, 
\end{align}
where the background object is completely defined. Moving to the perturbation part, it is required
that $k \gg \mathcal{H}$, meaning that gravitational wavelengths are smaller compared with the Hubble
scale, therefore, we will keep $\mathcal{H}$ at linear order, coupled to higher order of $k$.
\begin{align} 
    \delta b^{\mathcal{H}}_{0,0} & = 0 & \bar{b}^{\Lambda}_{0,0} & = 0 & 
    \delta \bar{b}^{\Lambda}_{0,0} & = 0
\end{align}

Using the above identification, we can derived the most general gravitational wave equation
whose parity symmetry es broken
\begin{align}
    & h''_{r,l} + h'_{r,l}\left(2\mathcal{H} + \lambda_{r,l}k^{n}
    \left(\frac{\alpha_{n}}{\Lambda^n a^n}\mathcal{H} + 
    \frac{\beta\left(\eta\right)}{\Lambda^{(n-1)}a^{(n-1)}}\right)\right) \notag \\ 
    & + k^2 h_{r,l}\left(1 + \lambda_{r,l}k^{m-1}
    \left(\frac{\gamma(\eta)}{\Lambda^m a^m}\mathcal{H}\right) 
    + \frac{\delta_{m}(\eta)}{\Lambda^{m-1}a^{m-1}}\right) = 0.
\end{align}
The general solution to the above equation is
\begin{equation}
    h_{r,l}(\eta)  = A_{r,l}(\eta)e^{-i\left(\phi(\eta) -k_{i}x^{i}\right)}, \\
\end{equation} 
ands its first and second derivatives can be compute directly from the above expression 
\begin{align}
    h'_{r,l}(\eta) & = -i\phi'(\eta)A_{r,l}(\eta)e^{-i\left(\phi(\eta) -k_{i}x^{i}\right)}, \\
    h''_{r,l}(\eta) & = -A_{r,l}(\eta)e^{-i\left(\phi(\eta) -k_{i}x^{i}\right)} 
    \left(i\phi''(\eta) + \phi'^{2}(\eta)\right),
\end{align} 
where it was assume that the amplitude of the gravitational wave varies 
on much longer timescales than the phase, therefore it is possible to neglected its 
time-conformal derivatives. Replacing the solution in Eq. \eqref{gw equation pv} leads to
\begin{align}
    & -\left(i\phi''(\eta) + \phi'^{2}(\eta)\right) -i\phi'(\eta)
    \left(2\mathcal{H} + \lambda_{r,l}k^{n}
    \left(\frac{\alpha_{n}}{\Lambda^n a^n}\mathcal{H} + 
    \frac{\beta\left(\eta\right)}{\Lambda^{(n-1)}a^{(n-1)}}\right)\right) \notag \\
    & k^2 \left(1 + \lambda_{r,l}k^{m-1}
    \left(\frac{\gamma(\eta)}{\Lambda^m a^m}\mathcal{H}\right) 
    + \frac{\delta_{m}(\eta)}{\Lambda^{m-1}a^{m-1}}\right) = 0.
\end{align}
which can be rewritten as
\begin{align}
    & \phi''(\eta) - i\phi'^{2}(\eta) + \phi'(\eta)
    \left(2\mathcal{H} + \lambda_{r,l}k^{n}
    \left(\frac{\alpha_{n}}{\Lambda^n a^n}\mathcal{H} + 
    \frac{\beta\left(\eta\right)}{\Lambda^{(n-1)}a^{(n-1)}}\right)\right) \notag \\
    & i k^2 \left(1 + \lambda_{r,l}k^{m-1}
    \left(\frac{\gamma(\eta)}{\Lambda^m a^m}\mathcal{H}\right) 
    + \frac{\delta_{m}(\eta)}{\Lambda^{m-1}a^{m-1}}\right) = 0.
\end{align}
Now, assuming that parity breaking terms are small deviations from general relativity,
it is possible to linearized the wave equation by taking
\begin{equation}
    \phi = \bar{\phi} + \delta \phi,
\end{equation}
where $\bar{\phi}$ is the usual background solution of general relativity which has 
the form of $\bar{\phi} ' = \pm k - i\mathcal{H}$.

\bibliographystyle{utphys}
\bibliography{references}

\end{document}