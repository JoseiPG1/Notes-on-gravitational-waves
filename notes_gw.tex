% type of document
\documentclass{article}
% new section new page
\usepackage{titlesec}
\newcommand{\sectionbreak}{\clearpage}
% hyperreferences
\usepackage{hyperref}
% align equation
\usepackage{amsmath}
% title
\title{Personal notes on gravitational waves}
% author
\author{Jose Perdiguero Garate}

\begin{document}
\maketitle
\tableofcontents

\section*{Notation and conventions}

I am working with the following conventions, for the metric tensor
\begin{equation}
    \left(-1, +1, +1, +1\right)
\end{equation}

\section{Linear general relativity}

In this section I am presenting a general overview on how to build-up
the Einstein's field equation, given a background metric tensor and a 
perturbation tensor.

\subsection{General overview}

Decomposed the metric tensor as the sum of a background metric, in this case 
the flat spacetime Minkowski metric $\eta_{\mu\nu}$,plus a perturbation 
$h_{\mu\nu}$ as follows
\begin{equation}
    \label{flat metric tensor}
    g_{\mu\nu} =  \eta_{\mu\nu} + h_{\mu\nu}.
\end{equation}
Using the definition of the Kronecker delta object it is straightforward to
obtain the inverse tensor of the perturbation
\begin{equation}
    \label{definition delta}
    \delta^{\mu}_{\nu} = g^{\mu\sigma}g_{\sigma\nu}.
\end{equation}
Replacing Eq.(\ref{flat metric tensor}) in to Eq.(\ref{definition delta}), and
neglecting second order terms in the perturbation field, leads to
\begin{equation}
    h^{\mu\nu} = -\eta^{\mu\alpha}\eta^{\nu\beta}h_{\alpha\beta},
\end{equation}
notice that, in order to upper/lower the indices of the perturbation I am only
using the background metric.\footnote{Including the perturbation tensor leads
to second order terms, which I am ignoring.} Schematically, second order terms
are neglected 
\begin{align}
    h_{\mu\nu}h_{\alpha\beta} & \sim 0, & 
    h_{\mu\nu}\partial_{\gamma} h_{\alpha\beta} & \sim 0, &
    \partial_{\delta} h_{\mu\nu}\partial_\gamma h_{\alpha\beta} & \sim 0.
\end{align}
This is the general overview of the fundamental field of general relativity,
which at its core is the metric tensor. In the following subsection, I will
be computing the Einstein's field equations for the metric tensor written 
in Eq.\ref{flat metric tensor}.

\section{Curved gravity}

\end{document}