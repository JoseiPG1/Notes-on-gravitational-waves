% type of document
\documentclass{article}
% new section new page
\usepackage{titlesec}
\newcommand{\sectionbreak}{\clearpage}
% hyperreferences
\usepackage{hyperref}
% align equation
\usepackage{amsmath}
% figures such as square
\usepackage{amssymb}
% title
\title{Personal notes on gravitational waves}
% author
\author{Jose Perdiguero Garate}

\begin{document}
\maketitle
\tableofcontents

\section*{Notation and conventions}

I am working with the following conventions, for the metric tensor
\begin{equation}
    \left(-1, +1, +1, +1\right)
\end{equation}

\section{Linear general relativity}

In this section I am presenting a general overview on how to build-up
the Einstein's field equation, given a background metric tensor and a 
perturbation tensor.

\subsection{General overview}

Decomposed the metric tensor as the sum of a background metric, in this case 
the flat spacetime Minkowski metric $\eta_{\mu\nu}$,plus a perturbation 
$h_{\mu\nu}$ as follows
\begin{equation}
    \label{metric tensor}
    g_{\mu\nu} =  \eta_{\mu\nu} + h_{\mu\nu}.
\end{equation}
Using the definition of the Kronecker delta object it is straightforward to
obtain the inverse tensor of the perturbation
\begin{equation}
    \label{definition delta}
    \delta^{\mu}_{\nu} = g^{\mu\sigma}g_{\sigma\nu}.
\end{equation}
Replacing Eq.\eqref{metric tensor} in to Eq.\eqref{definition delta}, and
neglecting second order terms in the perturbation field, leads to
\begin{equation}
    h^{\mu\nu} = -\eta^{\mu\alpha}\eta^{\nu\beta}h_{\alpha\beta},
\end{equation}
notice that, in order to upper/lower the indices of the perturbation I am only
using the background metric.\footnote{Including the perturbation tensor leads
to second order terms, which I am ignoring.} Schematically, second order terms
are neglected 
\begin{align}
    h_{\mu\nu}h_{\alpha\beta} & \sim 0, & 
    h_{\mu\nu}\partial_{\gamma} h_{\alpha\beta} & \sim 0, &
    \partial_{\delta} h_{\mu\nu}\partial_\gamma h_{\alpha\beta} & \sim 0.
\end{align}
This is the general overview of the fundamental field of general relativity,
which at its core is the metric tensor. In the following subsection, I will
be computing the Einstein's field equations for the metric tensor written 
in Eq.\eqref{metric tensor}.

\subsection{Geometrical objects}

The first object that is required to compute the Einstein's field equations is the
connection. Working on a torsion-free manifold, the Levi-Civita connection is written
as
\begin{equation}
    \label{levi civita connection}
    \Gamma^{\mu}{}_{\alpha\beta} = \frac{1}{2}g^{\mu\rho}\left(
        \partial_{\alpha}g_{\rho\beta} + \partial_{\beta}g_{\rho\alpha} -
        \partial_{\rho}g_{\alpha\beta}.
    \right)
\end{equation}
Replacing Eq.\eqref{metric tensor} in to Eq.\eqref{levi civita connection} leads to
\begin{equation}
    \Gamma^{\mu}{}_{\alpha\beta} = \frac{1}{2}\left(\eta^{\mu\rho} - h^{\mu\rho}\right)
        \left(\partial_{\alpha}\eta_{\rho\beta} 
        + \partial_{\beta}\eta_{\rho\alpha} 
        - \partial_{\rho}\eta_{\alpha\beta}
        + \partial_{\alpha}h_{\rho\beta} 
        + \partial_{\beta}h_{\rho\alpha} 
        - \partial_{\rho}h_{\alpha\beta}\right).
\end{equation}
Notice that, the only non-trivial contributions are the ones that are linear in the
perturbation, additionally, the partial derivatives of the Minkowski's metric tensor vanishes,
therefore, the connection coefficients are reduced to
\begin{equation}
    \label{levi civita connection flat + perturbation}
    \Gamma^{\mu}{}_{\alpha\beta} = \frac{1}{2}\eta^{\mu\rho}
        \left(\partial_{\alpha}h_{\rho\beta} + \partial_{\beta}h_{\rho\alpha} 
        - \partial_{\rho}h_{\alpha\beta}\right).
\end{equation}

Next, compute the Riemann curvature tensor
\begin{equation}
    \label{riemann curvature tensor}
    \mathcal{R}^{\rho}{}_{\sigma\mu\nu} = \partial_{\mu}\Gamma^{\rho}{}_{\sigma\nu}
    - \partial_{\nu}\Gamma^{\rho}{}_{\sigma\mu} 
    + \Gamma^{\gamma}{}_{\nu\sigma}\Gamma^{\rho}{}_{\mu\gamma}
    + \Gamma^{\gamma}{}_{\mu\sigma}\Gamma^{\rho}{}_{\nu\gamma},
\end{equation}
however, instead of computing directly from the above equation, it is convenient to 
notice the structure of the curvature tensor. The last two terms are quadratic in the 
Levi-Civita connection, and, since the connection is written with perturbation, then,
square terms in the connection vanishes, reducing the Riemann curvature tensor to
\begin{equation}
    \label{curvature tensor flat}
    \mathcal{R}^{\rho}{}_{\sigma\mu\nu} = \partial_{\mu}\Gamma^{\rho}{}_{\sigma\nu}
    - \partial_{\nu}\Gamma^{\rho}{}_{\sigma\mu}.
\end{equation}
Replacing Eq.\eqref{levi civita connection flat + perturbation} in to 
Eq.\eqref{curvature tensor flat} leads to
\begin{equation}
    \mathcal{R}^{\rho}{}_{\sigma\mu\nu} = \frac{1}{2}\eta^{\rho \alpha}\partial_{\mu}\left(
        \partial_{\sigma}h_{\alpha\nu} + \partial_{\nu}h_{\sigma\alpha}
        - \partial_{\alpha}h_{\sigma\nu}\right) +
        \frac{1}{2}\eta^{\rho \alpha}\partial_{\nu}\left(
        \partial_{\sigma}h_{\alpha\mu} + \partial_{\mu}h_{\sigma\alpha}
        - \partial_{\alpha}h_{\sigma\mu}\right),
\end{equation}
the above expression, can be simplified to
\begin{equation}
    \mathcal{R}^{\rho}{}_{\sigma\mu\nu} = \frac{1}{2}\eta^{\rho\alpha}\left(
        \partial_{\mu}\partial_{\sigma}h_{\alpha\nu} 
        - \partial_{\mu}\partial_{\alpha}h_{\sigma\nu}
        - \partial_{\nu}\partial_{\sigma}h_{\mu\alpha}
        + \partial_{\nu}\partial_{\alpha}h_{\mu\sigma}\right).
\end{equation}

From the Riemann tensor, it is straightforward to compute the Ricci tensor, by
contracting their respective indices
\begin{equation}
    \mathcal{R}_{\sigma\nu} = \mathcal{R}^{\mu}{}_{\sigma\mu\nu}.
\end{equation}
A direct computation shows the structure of the Ricci tensor
\begin{equation}
    \label{ricci tensor flat + perturbation}
    \mathcal{R}_{\sigma\nu} = \frac{1}{2}\left(
        \partial^{\alpha}\partial_{\sigma}h_{\alpha\nu} 
        -  \square h_{\sigma\nu} - \partial_{\nu}\partial_{\sigma}h
        + \partial_{\nu}\partial^{\alpha}h_{\alpha\sigma}\right),
\end{equation}
where $\square$ is the d'Alembert operator and $h$ is the trace of the perturbation.

In the same spirit, the curvature scalar can be obtained directly through
the contraction of the Ricci tensor
\begin{equation}
    \mathcal{R} = g^{\mu\sigma}\mathcal{R}_{\mu\sigma}.
\end{equation}
This computation is straightforward
\begin{equation}
    \mathcal{R} = \left(\eta^{\mu\sigma} - h^{\mu\sigma}\right) \frac{1}{2}\left(
        \partial^{\alpha}\partial_{\sigma}h_{\alpha\nu} 
        -  \square h_{\sigma\nu} - \partial_{\nu}\partial_{\sigma}h
        + \partial_{\nu}\partial^{\alpha}h_{\alpha\sigma}\right).
\end{equation}
Neglecting second order terms in the perturbation field, the scalar curvature is given
by
\begin{equation}
    \label{curvature scalar flat + perturbation}
    \mathcal{R} = \partial_{\mu}\partial_{\sigma}h^{\mu\sigma} - \square h
\end{equation}

Now, we can compute the Einstein's field equations without a cosmological constant
\begin{equation}
    \label{Einsteins field equation}
    \mathcal{R}_{\mu\nu} - \frac{1}{2}\mathcal{R}g_{\mu\nu} = \frac{8\pi G}{c^4}
    \mathcal{T}_{\mu\nu}
\end{equation}
where $\mathcal{T}_{\mu\nu}$ is the energy momentum tensor. Replacing 
Eq.\eqref{ricci tensor flat + perturbation} and Eq.\eqref{curvature scalar flat + perturbation}
in to Eq.\eqref{Einsteins field equation} leads to
\begin{equation}
        \partial^{\alpha}\partial_{\mu}h_{\alpha\nu} 
        -  \square h_{\mu\nu} - \partial_{\nu}\partial_{\mu}h
        + \partial_{\nu}\partial^{\alpha}h_{\alpha\mu}
        - \left(\partial_{\alpha}\partial_{\beta}h^{\alpha\beta} - \square h\right)
        \left(\eta_{\mu\nu} + h_{\mu\nu}\right) = \frac{16\pi G}{c^4}\mathcal{T}_{\mu\nu}.
\end{equation}
Just like before, neglecting second order terms in the perturbation, the above equation
can be reduced to
\begin{equation}
    \label{einsteins equation flat + perturbation}
        \partial^{\alpha}\partial_{\mu}h_{\alpha\nu} 
        -  \square h_{\mu\nu} - \partial_{\nu}\partial_{\mu}h
        + \partial_{\nu}\partial^{\alpha}h_{\alpha\mu}
        - \eta_{\mu\nu}\partial_{\alpha}\partial_{\beta}h^{\alpha\beta} 
        - \eta_{\mu\nu}\square h = \frac{16\pi G}{c^4}\mathcal{T}_{\mu\nu}.
\end{equation}
The above equation, can be written in a much more compact manner by using the following
variable change\footnote{In the standard literature $X_{\mu\nu}$ is written as $\bar{h}$,
but I strongly believed this leads to confunsions.}
\begin{equation}
    X_{\mu\nu} = h_{\mu\nu} - \frac{1}{2}\eta_{\mu\nu}h,
\end{equation}
which can be inverted through standard methods
\begin{equation}
    \label{variable change}
    h_{\mu\nu} = X_{\mu\nu} - \frac{1}{2}\eta_{\mu\nu}X
\end{equation}
where $X$ is the trace of the tensor $X_{\mu\nu}$, and the tensor $X_{\mu\nu}$, also satisfies
the relation $X = -h$. Replacing the variable change written in Eq.\eqref{variable change} in 
Eq.\eqref{einsteins equation flat + perturbation} and simplifying terms, leads to
\begin{equation}
    \partial^{\alpha}\partial_{\mu}X_{\alpha\nu} + \partial_{\nu}\partial^{\alpha}X_{\alpha\mu}
    - \square X_{\mu\nu} - \eta_{\mu\nu}\partial_{\alpha}\partial_{\beta}X^{\alpha\beta}
    =  \frac{16\Pi G}{c^4}\mathcal{T}_{\mu\nu},
\end{equation}
which, in some sense has a more simple structure that Eq.\eqref{einsteins equation flat + perturbation},
and also contains the wave operator. Nonetheless, this does not look like a gravitational 
wave equations. In the next subsection, I will show you, how can you derived the gravitational
wave equation from the above expression using a gauge transformation.



\end{document}